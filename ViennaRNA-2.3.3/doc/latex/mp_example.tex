\hypertarget{mp_example_example_c}{}\section{Example programs using the R\+N\+Alib C library}\label{mp_example_example_c}
\hypertarget{mp_example_examples_c_old_API}{}\subsection{Using the \textquotesingle{}old\textquotesingle{} A\+PI}\label{mp_example_examples_c_old_API}
The following program exercises most commonly used functions of the library. The program folds two sequences using both the mfe and partition function algorithms and calculates the tree edit and profile distance of the resulting structures and base pairing probabilities.

\begin{DoxyNote}{Note}
This program uses the old A\+PI of R\+N\+Alib, which is in part already marked deprecated. Please consult the \hyperlink{newAPI}{R\+N\+Alib A\+PI v3.\+0} page for details of what changes are necessary to port your implementation to the new A\+PI.
\end{DoxyNote}

\begin{DoxyCodeInclude}
\textcolor{preprocessor}{#include  <stdio.h>}
\textcolor{preprocessor}{#include  <stdlib.h>}
\textcolor{preprocessor}{#include  <math.h>}
\textcolor{preprocessor}{#include  <string.h>}
\textcolor{preprocessor}{#include  "\hyperlink{utils_8h}{utils.h}"}
\textcolor{preprocessor}{#include  "\hyperlink{fold__vars_8h}{fold\_vars.h}"}
\textcolor{preprocessor}{#include  "\hyperlink{fold_8h}{fold.h}"}
\textcolor{preprocessor}{#include  "\hyperlink{part__func_8h}{part\_func.h}"}
\textcolor{preprocessor}{#include  "\hyperlink{inverse_8h}{inverse.h}"}
\textcolor{preprocessor}{#include  "\hyperlink{RNAstruct_8h}{RNAstruct.h}"}
\textcolor{preprocessor}{#include  "\hyperlink{treedist_8h}{treedist.h}"}
\textcolor{preprocessor}{#include  "\hyperlink{stringdist_8h}{stringdist.h}"}
\textcolor{preprocessor}{#include  "\hyperlink{profiledist_8h}{profiledist.h}"}

\textcolor{keywordtype}{void} main()
\{
   \textcolor{keywordtype}{char} *seq1=\textcolor{stringliteral}{"CGCAGGGAUACCCGCG"}, *seq2=\textcolor{stringliteral}{"GCGCCCAUAGGGACGC"},
        *struct1,* struct2,* xstruc;
   \textcolor{keywordtype}{float} e1, e2, tree\_dist, string\_dist, profile\_dist, kT;
   \hyperlink{structTree}{Tree} *T1, *T2;
   \hyperlink{structswString}{swString} *S1, *S2;
   \textcolor{keywordtype}{float} *pf1, *pf2;
   \hyperlink{group__data__structures_ga31125aeace516926bf7f251f759b6126}{FLT\_OR\_DBL} *bppm;
   \textcolor{comment}{/* fold at 30C instead of the default 37C */}
   \hyperlink{group__model__details_gab4b11c8d9c758430960896bc3fe82ead}{temperature} = 30.;      \textcolor{comment}{/* must be set *before* initializing  */}

   \textcolor{comment}{/* allocate memory for structure and fold */}
   struct1 = (\textcolor{keywordtype}{char}* ) \hyperlink{utils_8h_ad7e1e137b3bf1f7108933d302a7f0177}{space}(\textcolor{keyword}{sizeof}(\textcolor{keywordtype}{char})*(strlen(seq1)+1));
   e1 =  \hyperlink{group__mfe__fold__single_gaadafcb0f140795ae62e5ca027e335a9b}{fold}(seq1, struct1);

   struct2 = (\textcolor{keywordtype}{char}* ) \hyperlink{utils_8h_ad7e1e137b3bf1f7108933d302a7f0177}{space}(\textcolor{keyword}{sizeof}(\textcolor{keywordtype}{char})*(strlen(seq2)+1));
   e2 =  \hyperlink{group__mfe__fold__single_gaadafcb0f140795ae62e5ca027e335a9b}{fold}(seq2, struct2);

   \hyperlink{group__mfe__fold__single_ga107fdfe5fd641868156bfd849f6866c7}{free\_arrays}();     \textcolor{comment}{/* free arrays used in fold() */}

   \textcolor{comment}{/* produce tree and string representations for comparison */}
   xstruc = \hyperlink{group__struct__utils_ga78d73cd54a068ef2812812771cdddc6f}{expand\_Full}(struct1);
   T1 = \hyperlink{treedist_8h_a08fe4d5afd385dce593b86eaf010c6e3}{make\_tree}(xstruc);
   S1 = \hyperlink{stringdist_8h_a3125991b3a403b3f89230474deb3f22e}{Make\_swString}(xstruc);
   free(xstruc);

   xstruc = \hyperlink{group__struct__utils_ga78d73cd54a068ef2812812771cdddc6f}{expand\_Full}(struct2);
   T2 = \hyperlink{treedist_8h_a08fe4d5afd385dce593b86eaf010c6e3}{make\_tree}(xstruc);
   S2 = \hyperlink{stringdist_8h_a3125991b3a403b3f89230474deb3f22e}{Make\_swString}(xstruc);
   free(xstruc);

   \textcolor{comment}{/* calculate tree edit distance and aligned structures with gaps */}
   \hyperlink{dist__vars_8h_aa03194c513af6b860e7b33e370b82bdb}{edit\_backtrack} = 1;
   tree\_dist = \hyperlink{treedist_8h_a3b21f1925f7071f46d93431a835217bb}{tree\_edit\_distance}(T1, T2);
   \hyperlink{treedist_8h_acbc1cb9bce582ea945e4a467c76a57aa}{free\_tree}(T1); \hyperlink{treedist_8h_acbc1cb9bce582ea945e4a467c76a57aa}{free\_tree}(T2);
   \hyperlink{group__struct__utils_ga1054c4477d53b31d79d4cb132100e87a}{unexpand\_aligned\_F}(\hyperlink{dist__vars_8h_ac1605fe3448ad0a0b809c4fb8f6a854a}{aligned\_line});
   printf(\textcolor{stringliteral}{"%s\(\backslash\)n%s  %3.2f\(\backslash\)n"}, \hyperlink{dist__vars_8h_ac1605fe3448ad0a0b809c4fb8f6a854a}{aligned\_line}[0], \hyperlink{dist__vars_8h_ac1605fe3448ad0a0b809c4fb8f6a854a}{aligned\_line}[1], tree\_dist);

   \textcolor{comment}{/* same thing using string edit (alignment) distance */}
   string\_dist = \hyperlink{stringdist_8h_a89e3c335ef17780576d7c0e713830db9}{string\_edit\_distance}(S1, S2);
   free(S1); free(S2);
   printf(\textcolor{stringliteral}{"%s  mfe=%5.2f\(\backslash\)n%s  mfe=%5.2f  dist=%3.2f\(\backslash\)n"},
          \hyperlink{dist__vars_8h_ac1605fe3448ad0a0b809c4fb8f6a854a}{aligned\_line}[0], e1, \hyperlink{dist__vars_8h_ac1605fe3448ad0a0b809c4fb8f6a854a}{aligned\_line}[1], e2, string\_dist);

   \textcolor{comment}{/* for longer sequences one should also set a scaling factor for}
\textcolor{comment}{      partition function folding, e.g: */}
   kT = (\hyperlink{group__model__details_gab4b11c8d9c758430960896bc3fe82ead}{temperature}+273.15)*1.98717/1000.;  \textcolor{comment}{/* kT in kcal/mol */}
   \hyperlink{group__model__details_gad3b22044065acc6dee0af68931b52cfd}{pf\_scale} = exp(-e1/kT/strlen(seq1));

   \textcolor{comment}{/* calculate partition function and base pair probabilities */}
   e1 = \hyperlink{group__pf__fold_gadc3db3d98742427e7001a7fd36ef28c2}{pf\_fold}(seq1, struct1);
   \textcolor{comment}{/* get the base pair probability matrix for the previous run of pf\_fold() */}
   bppm = \hyperlink{group__pf__fold_gac5ac7ee281aae1c5cc5898a841178073}{export\_bppm}();
   pf1 = \hyperlink{profiledist_8h_a3dff26e707a2a2e65a0f759caabde6e7}{Make\_bp\_profile\_bppm}(bppm, strlen(seq1));

   e2 = \hyperlink{group__pf__fold_gadc3db3d98742427e7001a7fd36ef28c2}{pf\_fold}(seq2, struct2);
   \textcolor{comment}{/* get the base pair probability matrix for the previous run of pf\_fold() */}
   bppm = \hyperlink{group__pf__fold_gac5ac7ee281aae1c5cc5898a841178073}{export\_bppm}();
   pf2 = \hyperlink{profiledist_8h_a3dff26e707a2a2e65a0f759caabde6e7}{Make\_bp\_profile\_bppm}(bppm, strlen(seq2));

   \hyperlink{group__pf__fold_gae73db3f49a94f0f72e067ecd12681dbd}{free\_pf\_arrays}();  \textcolor{comment}{/* free space allocated for pf\_fold() */}

   profile\_dist = \hyperlink{profiledist_8h_abe75e90e00a1e5dd8862944ed53dad5d}{profile\_edit\_distance}(pf1, pf2);
   printf(\textcolor{stringliteral}{"%s  free energy=%5.2f\(\backslash\)n%s  free energy=%5.2f  dist=%3.2f\(\backslash\)n"},
          \hyperlink{dist__vars_8h_ac1605fe3448ad0a0b809c4fb8f6a854a}{aligned\_line}[0], e1, \hyperlink{dist__vars_8h_ac1605fe3448ad0a0b809c4fb8f6a854a}{aligned\_line}[1], e2, profile\_dist);

   \hyperlink{profiledist_8h_a9b0b84a5a45761bf42d7c835dcdb3b85}{free\_profile}(pf1); \hyperlink{profiledist_8h_a9b0b84a5a45761bf42d7c835dcdb3b85}{free\_profile}(pf2);
\}
\end{DoxyCodeInclude}


In a typical Unix environment you would compile this program using\+: 
\begin{DoxyCode}
00001 cc $\{OPENMP\_CFLAGS\} -c example.c -I$\{hpath\}
\end{DoxyCode}
 and link using 
\begin{DoxyCode}
00001 cc $\{OPENMP\_CFLAGS\} -o example -L$\{lpath\} -lRNA -lm
\end{DoxyCode}
 where {\itshape \$\{hpath\}} and {\itshape \$\{lpath\}} point to the location of the header files and library, respectively.

\begin{DoxyNote}{Note}
As default, the R\+N\+Alib is compiled with build-\/in {\itshape Open\+MP} multithreading support. Thus, when linking your own object files to the library you have to pass the compiler specific {\itshape \$\{O\+P\+E\+N\+M\+P\+\_\+\+C\+F\+L\+A\+GS\}} (e.\+g. \textquotesingle{}-\/fopenmp\textquotesingle{} for {\bfseries gcc}) even if your code does not use openmp specific code. However, in that case the {\itshape Open\+MP} flags may be ommited when compiling example.\+c
\end{DoxyNote}
\hypertarget{mp_example_examples_c_new_API}{}\subsection{Using the \textquotesingle{}new\textquotesingle{} v3.\+0 A\+PI}\label{mp_example_examples_c_new_API}
The following is a simple program that computes the M\+FE, partition function, and centroid structure by making use of the v3.\+0 A\+PI. 
\begin{DoxyCodeInclude}
\textcolor{preprocessor}{#include <stdio.h>}
\textcolor{preprocessor}{#include <stdlib.h>}
\textcolor{preprocessor}{#include <string.h>}

\textcolor{preprocessor}{#include  <\hyperlink{data__structures_8h}{ViennaRNA/data\_structures.h}>}
\textcolor{preprocessor}{#include  <\hyperlink{params_8h}{ViennaRNA/params.h}>}
\textcolor{preprocessor}{#include  <\hyperlink{utils_8h}{ViennaRNA/utils.h}>}
\textcolor{preprocessor}{#include  <\hyperlink{eval_8h}{ViennaRNA/eval.h}>}
\textcolor{preprocessor}{#include  <\hyperlink{fold_8h}{ViennaRNA/fold.h}>}
\textcolor{preprocessor}{#include  <\hyperlink{part__func_8h}{ViennaRNA/part\_func.h}>}


\textcolor{keywordtype}{int} main(\textcolor{keywordtype}{int} argc, \textcolor{keywordtype}{char} *argv[])\{

  \textcolor{keywordtype}{char}  *seq = \textcolor{stringliteral}{"
      AGACGACAAGGUUGAAUCGCACCCACAGUCUAUGAGUCGGUGACAACAUUACGAAAGGCUGUAAAAUCAAUUAUUCACCACAGGGGGCCCCCGUGUCUAG"};
  \textcolor{keywordtype}{char}  *mfe\_structure = \hyperlink{group__utils_gaf37a0979367c977edfb9da6614eebe99}{vrna\_alloc}(\textcolor{keyword}{sizeof}(\textcolor{keywordtype}{char}) * (strlen(seq) + 1));
  \textcolor{keywordtype}{char}  *prob\_string   = \hyperlink{group__utils_gaf37a0979367c977edfb9da6614eebe99}{vrna\_alloc}(\textcolor{keyword}{sizeof}(\textcolor{keywordtype}{char}) * (strlen(seq) + 1));

  \textcolor{comment}{/* get a vrna\_fold\_compound with default settings */}
  \hyperlink{group__fold__compound_structvrna__fc__s}{vrna\_fold\_compound\_t} *vc = \hyperlink{group__fold__compound_ga6601d994ba32b11511b36f68b08403be}{vrna\_fold\_compound}(seq, NULL, 
      \hyperlink{group__fold__compound_gacea5b7ee6181c485f36e2afa0e9089e4}{VRNA\_OPTION\_DEFAULT});

  \textcolor{comment}{/* call MFE function */}
  \textcolor{keywordtype}{double} mfe = (double)\hyperlink{group__mfe__fold_gabd3b147371ccf25c577f88bbbaf159fd}{vrna\_mfe}(vc, mfe\_structure);

  printf(\textcolor{stringliteral}{"%s\(\backslash\)n%s (%6.2f)\(\backslash\)n"}, seq, mfe\_structure, mfe);

  \textcolor{comment}{/* rescale parameters for Boltzmann factors */}
  \hyperlink{group__energy__parameters_gad607bc3a5b5da16400e2ca4ea5560233}{vrna\_exp\_params\_rescale}(vc, &mfe);

  \textcolor{comment}{/* call PF function */}
  \hyperlink{group__data__structures_ga31125aeace516926bf7f251f759b6126}{FLT\_OR\_DBL} en = \hyperlink{group__pf__fold_ga29e256d688ad221b78d37f427e0e99bc}{vrna\_pf}(vc, prob\_string);

  \textcolor{comment}{/* print probability string and free energy of ensemble */}
  printf(\textcolor{stringliteral}{"%s (%6.2f)\(\backslash\)n"}, prob\_string, en);

  \textcolor{comment}{/* compute centroid structure */}
  \textcolor{keywordtype}{double} dist;
  \textcolor{keywordtype}{char} *cent = \hyperlink{group__centroid__fold_ga0e64bb67e51963dc71cbd4d30b80a018}{vrna\_centroid}(vc, &dist);

  \textcolor{comment}{/* print centroid structure, its free energy and mean distance to the ensemble */}
  printf(\textcolor{stringliteral}{"%s (%6.2f d=%6.2f)\(\backslash\)n"}, cent, \hyperlink{group__eval_ga58f199f1438d794a265f3b27fc8ea631}{vrna\_eval\_structure}(vc, cent), dist);

  \textcolor{comment}{/* free centroid structure */}
  free(cent);

  \textcolor{comment}{/* free pseudo dot-bracket probability string */}
  free(prob\_string);

  \textcolor{comment}{/* free mfe structure */}
  free(mfe\_structure);

  \textcolor{comment}{/* free memory occupied by vrna\_fold\_compound */}
  \hyperlink{group__fold__compound_gadded6039d63f5d6509836e20321534ad}{vrna\_fold\_compound\_free}(vc);

  \textcolor{keywordflow}{return} EXIT\_SUCCESS;
\}
\end{DoxyCodeInclude}
\hypertarget{mp_example_scripting_perl_examples}{}\section{Perl Examples}\label{mp_example_scripting_perl_examples}
\hypertarget{mp_example_scripting_perl_examples_flat}{}\subsection{Using the Flat Interface}\label{mp_example_scripting_perl_examples_flat}
Example 1\+: \char`\"{}\+Simple M\+F\+E prediction\char`\"{} 
\begin{DoxyCodeInclude}
00001 #!/usr/bin/perl
00002 
00003 use warnings;
00004 use strict;
00005 
00006 use RNA;
00007 
00008 my $seq1 = "CGCAGGGAUACCCGCG";
00009 
00010 # compute minimum free energy (mfe) and corresponding structure
00011 my ($ss, $mfe) = RNA::fold($seq1);
00012 
00013 # print output
00014 printf "%s [ %6.2f ]\(\backslash\)n", $ss, $mfe;
\end{DoxyCodeInclude}
\hypertarget{mp_example_scripting_perl_examples_oo}{}\subsection{Using the Object Oriented (\+O\+O) Interface}\label{mp_example_scripting_perl_examples_oo}
The \textquotesingle{}fold\+\_\+compound\textquotesingle{} class that serves as an object oriented interface for \hyperlink{group__fold__compound_ga1b0cef17fd40466cef5968eaeeff6166}{vrna\+\_\+fold\+\_\+compound\+\_\+t}

Example 1\+: \char`\"{}\+Simple M\+F\+E prediction\char`\"{} 
\begin{DoxyCodeInclude}
00001 #!/usr/bin/perl
00002 
00003 use warnings;
00004 use strict;
00005 
00006 use RNA;
00007 
00008 my $seq1 = "CGCAGGGAUACCCGCG";
00009 
00010 # create new fold\_compound object
00011 my $fc = new RNA::fold\_compound($seq1);
00012 
00013 # compute minimum free energy (mfe) and corresponding structure
00014 my ($ss, $mfe) = $fc->mfe();
00015 
00016 # print output
00017 printf "%s [ %6.2f ]\(\backslash\)n", $ss, $mfe;
\end{DoxyCodeInclude}
\hypertarget{mp_example_scripting_python_examples}{}\section{Python Examples}\label{mp_example_scripting_python_examples}
\hypertarget{mp_example_scripting_python_examples_flat}{}\subsection{Using the Flat Interface}\label{mp_example_scripting_python_examples_flat}
Example 1\+: \char`\"{}\+Simple M\+F\+E prediction\char`\"{} 
\begin{DoxyCodeInclude}
00001 \textcolor{keyword}{import} RNA
00002 
00003 sequence = \textcolor{stringliteral}{"CUCGUCGCCUUAAUCCAGUGCGGGCGCUAGACAUCUAGUUAUCGCCGCAA"}
00004 
00005 \textcolor{comment}{# compute minimum free energy (mfe) and corresponding structure}
00006 (structure, mfe) = RNA.fold(sequence)
00007 
00008 \textcolor{comment}{# print output}
00009 \textcolor{keywordflow}{print} \textcolor{stringliteral}{"%s\(\backslash\)n%s [ %6.2f ]"} % (structure, mfe)
\end{DoxyCodeInclude}
\hypertarget{mp_example_scripting_python_examples_oo}{}\subsection{Using the Object Oriented (\+O\+O) Interface}\label{mp_example_scripting_python_examples_oo}
The \textquotesingle{}fold\+\_\+compound\textquotesingle{} class that serves as an object oriented interface for \hyperlink{group__fold__compound_ga1b0cef17fd40466cef5968eaeeff6166}{vrna\+\_\+fold\+\_\+compound\+\_\+t}

Example 1\+: \char`\"{}\+Simple M\+F\+E prediction\char`\"{} 
\begin{DoxyCodeInclude}
00001 \textcolor{keyword}{import} RNA;
00002 
00003 sequence = \textcolor{stringliteral}{"CGCAGGGAUACCCGCG"}
00004 
00005 \textcolor{comment}{# create new fold\_compound object}
00006 fc = RNA.fold\_compound(sequence)
00007 
00008 \textcolor{comment}{# compute minimum free energy (mfe) and corresponding structure}
00009 (ss, mfe) = fc.mfe()
00010 
00011 \textcolor{comment}{# print output}
00012 \textcolor{keywordflow}{print} \textcolor{stringliteral}{"%s [ %6.2f ]"} % (ss, mfe)
\end{DoxyCodeInclude}


Example 2\+: \char`\"{}\+Use callback while generating suboptimal structures\char`\"{} 
\begin{DoxyCodeInclude}
00001 \textcolor{keyword}{import} RNA
00002 
00003 sequence = \textcolor{stringliteral}{"GGGGAAAACCCC"}
00004 
00005 \textcolor{comment}{# Set global switch for unique ML decomposition}
00006 RNA.cvar.uniq\_ML = 1
00007 
00008 subopt\_data = \{ \textcolor{stringliteral}{'counter'} : 1, \textcolor{stringliteral}{'sequence'} : sequence \}
00009 
00010 \textcolor{comment}{# Print a subopt result as FASTA record}
00011 \textcolor{keyword}{def }print\_subopt\_result(structure, energy, data):
00012     \textcolor{keywordflow}{if} \textcolor{keywordflow}{not} structure == \textcolor{keywordtype}{None}:
00013         \textcolor{keywordflow}{print} \textcolor{stringliteral}{">subopt %d"} % data[\textcolor{stringliteral}{'counter'}]
00014         \textcolor{keywordflow}{print} \textcolor{stringliteral}{"%s"} % data[\textcolor{stringliteral}{'sequence'}]
00015         \textcolor{keywordflow}{print} \textcolor{stringliteral}{"%s [%6.2f]"} % (structure, energy)
00016         \textcolor{comment}{# increase structure counter}
00017         data[\textcolor{stringliteral}{'counter'}] = data[\textcolor{stringliteral}{'counter'}] + 1
00018 
00019 \textcolor{comment}{# Create a 'fold\_compound' for our sequence}
00020 a = RNA.fold\_compound(sequence)
00021 
00022 \textcolor{comment}{# Enumerate all structures 500 dacal/mol = 5 kcal/mol arround}
00023 \textcolor{comment}{# the MFE and print each structure using the function above}
00024 a.subopt\_cb(500, print\_subopt\_result, subopt\_data);
\end{DoxyCodeInclude}


Example 3\+: \char`\"{}\+Revert M\+F\+E to Maximum Matching using soft constraint callbacks\char`\"{} 
\begin{DoxyCodeInclude}
00001 \textcolor{keyword}{import} RNA
00002 
00003 seq1 = \textcolor{stringliteral}{"CUCGUCGCCUUAAUCCAGUGCGGGCGCUAGACAUCUAGUUAUCGCCGCAA"}
00004 
00005 \textcolor{comment}{# Turn-off dangles globally}
00006 RNA.cvar.dangles = 0
00007 
00008 \textcolor{comment}{# Data structure that will be passed to our MaximumMatching() callback with two components:}
00009 \textcolor{comment}{# 1. a 'dummy' fold\_compound to evaluate loop energies w/o constraints, 2. a fresh set of energy parameters}
00010 mm\_data = \{ \textcolor{stringliteral}{'dummy'}: RNA.fold\_compound(seq1), \textcolor{stringliteral}{'params'}: RNA.param() \}
00011 
00012 \textcolor{comment}{# Nearest Neighbor Parameter reversal functions}
00013 revert\_NN = \{ 
00014     RNA.DECOMP\_PAIR\_HP:       \textcolor{keyword}{lambda} i, j, k, l, f, p: - f.eval\_hp\_loop(i, j) - 100,
00015     RNA.DECOMP\_PAIR\_IL:       \textcolor{keyword}{lambda} i, j, k, l, f, p: - f.eval\_int\_loop(i, j, k, l) - 100,
00016     RNA.DECOMP\_PAIR\_ML:       \textcolor{keyword}{lambda} i, j, k, l, f, p: - p.MLclosing - p.MLintern[0] - (j - i - k + l - 2) 
      * p.MLbase - 100,
00017     RNA.DECOMP\_ML\_ML\_STEM:    \textcolor{keyword}{lambda} i, j, k, l, f, p: - p.MLintern[0] - (l - k - 1) * p.MLbase,
00018     RNA.DECOMP\_ML\_STEM:       \textcolor{keyword}{lambda} i, j, k, l, f, p: - p.MLintern[0] - (j - i - k + l) * p.MLbase,
00019     RNA.DECOMP\_ML\_ML:         \textcolor{keyword}{lambda} i, j, k, l, f, p: - (j - i - k + l) * p.MLbase,
00020     RNA.DECOMP\_ML\_UP:         \textcolor{keyword}{lambda} i, j, k, l, f, p: - (j - i + 1) * p.MLbase,
00021     RNA.DECOMP\_EXT\_STEM:      \textcolor{keyword}{lambda} i, j, k, l, f, p: - f.E\_ext\_loop(k, l),
00022     RNA.DECOMP\_EXT\_STEM\_EXT:  \textcolor{keyword}{lambda} i, j, k, l, f, p: - f.E\_ext\_loop(i, k),
00023     RNA.DECOMP\_EXT\_EXT\_STEM:  \textcolor{keyword}{lambda} i, j, k, l, f, p: - f.E\_ext\_loop(l, j),
00024     RNA.DECOMP\_EXT\_EXT\_STEM1: \textcolor{keyword}{lambda} i, j, k, l, f, p: - f.E\_ext\_loop(l, j-1),
00025             \}
00026 
00027 \textcolor{comment}{# Maximum Matching callback function (will be called by RNAlib in each decomposition step)}
00028 \textcolor{keyword}{def }MaximumMatching(i, j, k, l, d, data):
00029     \textcolor{keywordflow}{return} revert\_NN[d](i, j, k, l, data[\textcolor{stringliteral}{'dummy'}], data[\textcolor{stringliteral}{'params'}])
00030 
00031 \textcolor{comment}{# Create a 'fold\_compound' for our sequence}
00032 fc = RNA.fold\_compound(seq1)
00033 
00034 \textcolor{comment}{# Add maximum matching soft-constraints}
00035 fc.sc\_add\_f(MaximumMatching)
00036 fc.sc\_add\_data(mm\_data, \textcolor{keywordtype}{None})
00037 
00038 \textcolor{comment}{# Call MFE algorithm}
00039 (s, mm) = fc.mfe()
00040 
00041 \textcolor{comment}{# print result}
00042 \textcolor{keywordflow}{print} \textcolor{stringliteral}{"%s\(\backslash\)n%s (MM: %d)\(\backslash\)n"} %  (seq1, s, -mm)
00043 
\end{DoxyCodeInclude}
 